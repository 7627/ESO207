\documentclass{article}
\usepackage[utf8]{inputenc}
\usepackage{listings}
\usepackage{color}
\usepackage{mathtools}
\usepackage{graphicx}
\graphicspath{ {images/} }

\definecolor{codegreen}{rgb}{0,0.6,0}
\definecolor{codegray}{rgb}{0.5,0.5,0.5}
\definecolor{codepurple}{rgb}{0.58,0,0.82}
\definecolor{backcolour}{RGB}{245,245,245}

\lstdefinestyle{mystyle}{
    %backgroundcolor=\color{backcolour},
    commentstyle=\color{codegreen},
    keywordstyle=\color{magenta},
    numberstyle=\tiny\color{codegray},
    stringstyle=\color{codepurple},
    basicstyle=\footnotesize,
    breakatwhitespace=false,
    breaklines=true,
    captionpos=b,
    keepspaces=true,
    numbers=left,
    numbersep=5pt,
    showspaces=false,
    showstringspaces=false,
    showtabs=false,
    tabsize=2
}

\lstset{style=mystyle}

\title{Theoretical Assignment 3}
\author{Amit Yadav }
\date{March 2018}

\begin{document}

\maketitle

\section*{Question 1}
  \textbf{Given:}
  \begin{itemize}
      \item A min heap,number of elements in heap, a number x, a number k $(k<n)$
  \end{itemize}
  \textbf{Algorithm:}\\
  \begin{lstlisting}[language=C]
    function(heap_array, n, x, k, index){
      static int i=0;
      if(i>=k) return 1; // 1 represent x is smaller than k_th smallest element
      if(index>n) return 0; //reached the end
      if(heap_array[index]<x){
         i++;
         return function(heap_array, n, x, k, 2*index) | function(heap_array, n, x, k, 2*index+1);
      }
      else return 0;
    }

    void main(){
      /* heap_array = [0,a1,a2,a3,a4,...,an] */
      function(heap_array, n, x, k, 1);
    }
  \end{lstlisting}
  \textbf{Explanation:}\\
  \textbf{Static int i} counts number of elements less than k. The moment $i\geq k$, our function returns 1.\\
  \break
  \textbf{Time complexity:}\\
  It take k+t max number of comparisions, where t is a number dependent upon heap and k. Therefore, \\
  \begin{center} Time = O(k) \end{center}
  \begin{center} Space = O(1) \end{center}
  \break

\section*{Question 2}
  \subsection*{Part 1}
    \textbf{Aim:} To find if an undirected graph is acyclic or not.\\
    Graph is given as a Adjacency list. As known, DFS visit takes O(V+E) time.
    I will do the same DFS visit but it will halt execution as soon as a cycle is found.
    Since, maximum length of a cycle can be $\mid V \mid$, so it will take max O(V) time.\\
    Worst case will be when it visits all the vertices and edges, which is only possible in case of no cycles present.
    In that case $\mid E \mid \leq \mid V-1 \mid$, equality holds when graph is connected. So, time will be O(2V) = O(V).\\
    \textbf{Algorithm:}
    \begin{lstlisting}[language=Python]
      cycleFound=False
      DFS-visit(v, Adj){
        for u in Adj[v]:
          if(cycleFound==True):
            return;
          if(u!=parent[v]):
            if(mark[u]==1):
              cycleFound=True; """If it visits a vertex which is already in process, means cycle is found"""
              return;
            else:
              parent[u]=v;
              mark[u]=1;
              DFS-visit(u, Adj);
              mark[u]=0;
      }

      DFS(Adj){
        for v in V:
          if v not in parent:
            if(cycleFound==True):
              return;
            parent[v]=None;
            mark[v]=1;
            DFS-visit(v, Adj);
            mark[v]=0;
      }

      DFS(Adj); """calling DFS function"""
    \end{lstlisting}

  \subsection*{Part 2}
    \textbf{Aim:} To find if a given undirected graph is tree or not. \\
    An undirected acyclic graph is a tree if it is connected. So, we essentially need to check if it is acyclic or not, and additionally,
    if it is connected or not. So our Algorithm will be almost same as \textbf{Part 1}, except there will be one additional condition.\\
    \textbf{Algorithm:}
    \begin{lstlisting}[language=Python]
      cycleFound=False
      Tree=True

      DFS-visit(v, Adj){
        for u in Adj[v]:
          if(cycleFound==True):
            return;
          if(u!=parent[v]):
            if(mark[u]==1):
              cycleFound=True; """If it visits a vertex which is already in process, means cycle is found"""
              return;
            else:
              parent[u]=v;
              mark[u]=1;
              DFS-visit(u, Adj);
              mark[u]=0;
      }

      DFS(Adj){
        parent[v]=None;
        mark[v]=1;
        DFS-visit(v, Adj);
        mark[v]=0;
        for v in V: """We run DFS-visit on any of the vertex and all the vertices should have been visited if it is a tree"""
          if v not in parent:
            Tree=False;
      }

      DFS(Adj); """calling DFS function"""
    \end{lstlisting}


\section*{Question 3}
  First we build a undirected graph with given words as vertices and two vertiices are connected if one is \textbf{allowed edit} of other.
  And then we run \textbf{BFS on word A}. \\
  So first we need to build the graph from a given D.
  \begin{lstlisting}[language=Python]
    allowed_edit(word1,word2){
      diff=0;
      size_diff=abs(len(word1)-len(word2));
      if size_diff>1:
        return 0;
      for i in range(0,min( len(word1), len(word1) )):
        if[word1[i]!=word2[i]]:
            diff+=1;
      if(size_diff==0 and diff==1):
        return 1;
      if(size_diff==1 and diff==0):
        return 1;
      return 0;
    }

    create_graph{
      for i in range(0, len(D)):
        for j in range(i+1, len(D)):
          if allowed_edit(D[i],D[j]):
            Adj[D[i]].append(D[j]);
            Adj[D[j]].append(D[i]);
    }

  \end{lstlisting}

  Building graph takes $O(n^2c^2)$ time.\\
  Then running time of BFS on given word is $O(V+E)$, where $\mid V \mid = len(D)$.
  \begin{lstlisting}[language=Python]
    BFS(B, Adj){
      level={B:0};
      parent={B:None}
      i=1;
      x=[B]
      while(x):
        neighbours=[];
        for u in x:
          for v in Adj[u]:
            if v not in level:
              level[v]=i;
              parent[v]=u;
              neighbours.append(v);
              if(v==A): """stop when we reach word A"""
                x=[];
                break;
          x=neighbours;
          i+=1;

      while(A!=None):
        print(A);
        A=parent[A];
    }
  \end{lstlisting}
\textbf{Time Complexity:}
\begin{enumerate}
  \item Creating Graph: \\
    Since, it iterates over D for every word i.e $O(n^2)$, and every check of allowed edit take $O(c^2)$ time,
    so total time complexity of creating Adjacency list is $O(n^2c^2)$.
  \item BFS on word B:\\
    Since BFS takes O(V+E) time, in worst case, $\mid E \mid=O(\mid V \mid^2)$. So, BFS take $O(n^2)$ time.
\end{enumerate}
Therefore total time can be expressed as $O(n^2c^2)$.\\
This is also the worst case time complexity of the algorithm.

\section*{Question 4}
\textbf{Given:} $h(x)= x mod 10$, a hash table with some entries.\\
\textbf{To find:} Number of ways to create this given hash table.\\
\textbf{Sol.} \textbf{42, 23, 34} and \textbf{46} are at their excpected places but \textbf{52} and \textbf{33} are not.
So it is clear that 42, 23 and 34 must be inserted before 52, else 52 would have occupied one of their places.
Also, 33 must be inserted after 23, 34, 52 and 46 because of the same reason. Since 52 is inserted before 33 implies 42 is also
insertes before 33, making 33 to be inserted in the last.
So possible orders of insertion are:\\
\underbrace{{42, 23, 34}}_\text{$3!$}, 52, 46, 33\\
\underbrace{{42,23, 34, 46}}_\text{$4!$}, 52, 33\\
Therefore, total number of ways= $3! + 4! = 30$


\section*{Question 5:}
Spanning tree with maximum height will look like this:\\
\includegraphics[width=\textwidth, height=8cm]{spanning_tree.jpg}\\
When we run DFS, we keep only the tree edges and no backward edges which lead to cycles. So, the result will be this spanning tree.\\
Height of the tree is \textbf{18}.






\end{document}
